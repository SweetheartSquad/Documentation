\section{Character Interactions}
While exploring, the player may focus on an NPC by looking at them from within a certain distance. This will present the player with two options: Talk and Diss. These can be seen in Figure~\ref{fig:UI_npc_talk}.

\subsection{Talk: Conversation}
\label{sec:conversation}
There are two types of conversations the player can have with NPCs in \ourgame{}.

The first type of conversation will be those assigned to only plot-specific characters. While conversing with a plot-specific character, player movement will be disabled and the camera will focus on the NPC. As the character speaks, dialogue is shown in a speech bubble at the top of the screen. These characters will usually have more than one line of dialogue, so the player may manually advance the conversation at any time. During the conversation, the player may occasionally be prompted to choose something to say from a list of dialogue choices. Each choice will be shown as a speech bubble at the bottom of the screen, as shown in Figure~\ref{fig:dialogue}. When a speech bubble is chosen, the conversation will progress according to the player's choice. All of these described conversation elements will be written in the scenario files, and may include triggers for changing player state, affecting the game world, or initiating the DISS battles described in Section~\ref{sec:yelling_contest}.

The second type of conversation triggers when the focused NPC is in a \textbf{non-speaking} state. While in a non-speaking state, the character will speak a single line, shown as text in a speech bubble above their head. This can be seen in Figure~\ref{fig:speech_bubble}. Characters are considered to be in a non-speaking state if they are not involved with any plot events, if they are upset with the player, or they are busy with a particular action. This line will be procedurally generated using a list of phrase structures and terms. In order to provide variety, three lines will exist for a single character and will be cycled through on multiple interactions. The lines will be different based on whether the player has DISSed the character or not.

\begin{figure}[H]
  \centering\begin{subfigure}{.45\textwidth}
    \centering
    \includegraphics[width=.9\linewidth]{images/UI_npc_nodialogue}
    \caption{Character without dialogue choices}
	\label{fig:speech_bubble}
  \end{subfigure}%
  \begin{subfigure}{.45\textwidth}
    \centering
    \includegraphics[width=.9\linewidth]{images/UI_npc_conversation}
    \caption{Character with dialogue choices}
	\label{fig:dialogue}
  \end{subfigure}%
  \caption{Characters reacting to talk interactions}
\end{figure}

\subsection{DISS: DISS Battles}
\label{sec:yelling_contest}

DISS battles may be initiated in one of two ways: the player may trigger one by intentionally selecting the "Diss" option while interacting with a character, or they may be triggered by NPCs as part of a scripted sequence. The interaction of a DISS battle can be roughly described as a tug-of-war between the two sides of the conversation, wherein a victor is decided when one side has obliterated the opponent's \textbf{self-confidence}, represented by a shared health bar at the top of the screen.

When an NPC is in \textbf{insult} mode, they will bombard the player with continuous insults, each one damaging the player's self-confidence. This can be seen in Figure~\ref{fig:yelling_contest_a}. In order to gain control, the player is given the opportunity to \textbf{interject} when the NPC pauses at the end of a thought or a sentence. If the player successfully interjects the opposing NPC mid-insult, shown in Figure~\ref{fig:yelling_contest_b}, they will be put into insult mode. While in insult mode, the player will be presented a partial insult with two possible conclusions: One of the presented options will complete the insult, and the other will result in an ineffective insult; an example is given in Figure~\ref{fig:yelling_contest_c}. The player has a limited time to choose the correct conclusion. If the player chooses incorrectly or their time runs out, the opposing NPC will attempt to interject and regain control. If the player chooses correctly, they will tip the scale towards their victory and be presented with another partial insult to complete. The difficulty will increase linearly with the time spent in insult mode by reducing the length of time given to the player to complete insults.

\begin{figure}[H]
  \centering\begin{subfigure}{.44\textwidth}
    \centering
    \includegraphics[width=.9\linewidth]{images/UI_yelling_defense}
    \caption{NPC in control - player attempts to interject at highlighted pauses}
    \label{fig:yelling_contest_a}
  \end{subfigure}%
  \begin{subfigure}{.44\textwidth}
    \centering
    \includegraphics[width=.9\linewidth]{images/UI_yelling_interrupt}
    \caption{Successful interjection by player}
    \label{fig:yelling_contest_b}
  \end{subfigure}
  
  \begin{subfigure}{.44\textwidth}
    \centering
    \includegraphics[width=.9\linewidth]{images/UI_yelling_offense}
    \caption{Player in control - player chooses insults instead of compliments while maintaining the rhythm}
    \label{fig:yelling_contest_c}
  \end{subfigure} %
	\begin{subfigure}{.44\textwidth}
	  \centering
	  \includegraphics[width=.9\linewidth]{images/UI_yelling_lifelost}
	  \caption{Player with a lost life}
	  \label{fig:yelling_contest_c}
	\end{subfigure}%
  \caption{DISS battle mockups}
  \label{fig:yelling_contest}
\end{figure}

\subsubsection{Friendship Tokens}
In \ourgame{}, the player can collect friendship tokens from NPCs that act as lives during DISS battles. These lives are represented by portraits of each NPC, found in the top-left corner of the screen. The player can earn an NPCs friendship by completing certain quests. If the player loses the round, the portrait of an NPC will be crossed out to show that they have been used up. If the player loses a round when all member of their squad have been used, they will lose the contest. Upon starting a new DISS battle, any tokens that were used in previous matches will be usable again.

\subsubsection{D.I.S.S.}
Each character in \ourgame{} will have four primary statistics which affect their performance in DISS battles. Together, these form the D.I.S.S. system. When in DISS battles, the statistics of the active participants will heavily influence the player's ability to win.

\begin{description}
\item[Defense]{decreases damage to the character's self-confidence}
\item[Insight]{makes the correct choice more apparent than the other}
\item[Strength]{increases damage to other character's self-confidence}
\item[Sass]{how long you have to make the correct input (affects how long opponents gaps are)}
\end{description}