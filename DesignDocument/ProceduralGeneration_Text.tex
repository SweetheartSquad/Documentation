\clearpage
\section{Generative Text}

\subsection{Incidental Dialogue}
\label{sec:incidental_dialogue}
In order to add dialogue to incidental characters, a generative sentence structure will be developed. This system will randomly select from a set of skeleton sentences. Each skeleton sentence will have a set of inputs which will be filled using word sets. Word sets will be broken down grammatically into sections such as nouns, adjectives, verbs, and adverbs. The skeleton sentence will specify which type of word it requires for each slot. An example of a skeleton sentence would be \verb|"Have you seen the {adjective} {noun}?"|. Each word set will be comprised  of more than one hundred words adding a huge amount of variability to player-character interactions. Skeleton sentences will also have placeholders for Character names. This will allow sentences to refer to characters from the scenarios that were loaded for the current run. An example of this would be "\verb|"I heard {character} kissed {character}"|. This will result in incidental dialogue being more relevant. 

\subsection{Insults}
Similar to incidental dialogue, described in Section~\ref{sec:incidental_dialogue}, skeleton sentences  will be used to create dynamic insults. Insults will be used exclusively in yelling contests and will be spoken by both the player and opposing character. An example insult would be \verb|"{exclamation}! Your {noun} looks like some sort of {adjective}{noun}!"|. The insult word sets and skeleton sentences would be different than those used for the incidental dialogue as the content for these two system will be very different. Procedural insults will make the yelling matches more entertaining as the player will constantly be presented with new word combinations, preventing them from becoming bored. 