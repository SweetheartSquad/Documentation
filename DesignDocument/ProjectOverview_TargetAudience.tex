\section{Target Audience}
\ourteam{}'s target audience for this project will primarily be of the 18-29 age group. Reports from the PEW research center indicates that high percentages of this demographic are video game players. In most countries, this age group also encompasses university students; there is a good chance that student users will have had party experiences of their own to draw upon while playing \ourgame{}.

\ourgame{} will be a surreal single-player game in a first-person perspective. Games that use this formula, such as \textit{Crypt Worlds}, \textit{Slave of God}, and \textit{Proteus}, have found a dedicated audience by emphasizing exploration and discovery over more traditional game mechanics. \textit{Crypt Worlds} and \textit{Slave of God} have been featured in publications such as PC Gamer, Kill Screen Daily, Rock Paper Shotgun, and IndieGames.com. \textit{Proteus} has been nominated for and won numerous independent game awards, and was recently featured in the 2012 exhibit "Common Senses" at the Museum of Modern Art.

The story of \ourgame{} is based around exploration and discovery. The player searches for clues to solve the larger mystery of Omar Clean's identity. Each clue reveals a small part of the narrative through anecdotes and pieces of Omar's past. This style of environmental, player-driven narrative has seen critical success in the current video game market. In February 2014, The Fullbright Company, creators of acclaimed story-rich adventure \textit{Gone Home}, announced that their game had sold over 250,000 copies. This game was praised for the way the player explores the environment to discover clues that uncover a mystery about the disappearance game's characters. \ourgame{} will leverage this basic framework and apply it to a new fictional world.

The cultural trend that drives professional and independent developers toward comedic and experimental forms of storytelling finds inspiration in the past. \ourgame{} draws inspiration from classic adventure game titles such as Double Fine's \textit{Full Throttle} and  \textit{Psychonauts}, which featured exagerrated characters and funny, sometimes mind-bending stories. The latter title featured an unconventional and humorous story about travelling into other people's minds, but was considered a commercial failure at release. Since then, it has developed a large cult following -- as of March 2012, has sold over 400,000 retail copies, not including sales from online platforms such as Steam and GOG. \ourgame{} seeks to target this audience by providing cartoonish characters, and a humourous story that takes them across many planes of reality.

In order to attract the attention of the target demographic, \ourteam{} will be creating a game that follows this current trend of experimental interactive experiences and provides the audience with unconventional gameplay and humour. Like the games mentioned above, it will appeal to an audience who, frustrated with the restrictive ideals of the big budget game industry, is looking for unique narrative experiences.