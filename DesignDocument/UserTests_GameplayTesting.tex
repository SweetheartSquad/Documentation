\clearpage
\section{Gameplay Tests}

Throughout development it will be necessary to test the different gameplay mechanics incorporated in \ourgame{} to ensure that players are not confused or frustrated while trying to play. During these tests, subjective and objective data will be collected. Subjective data will be obtained through the use of Likert scale questions, where users respond to a statement with a number between 1 and 5, where 1 represents Strongly Disagree and 5 represents Strongly Agree, as well as allotted space for testers to further explain their opinions.

\subsection{Yelling Mechanic}
The yelling mechanic described in Section~\ref{sec:yelling_contest} is a fairly complex, self-contained element of \ourgame{}. As such, tests will have to be performed to check if the controls are responsive and if the difficulty scales properly as the player progresses through the game. To test this, users will be asked to play the game at multiple difficulties while being discreetly monitored. Data will be collected on the following objective metrics:

\subsubsection{Interrupt Section:}
\begin{itemize}
\item{The percentage of interjections that fail}
\item{The average duration between an interjection and the target interjection time}
\item{The percentage of times an incorrect key is pressed}
\end{itemize}
\subsubsection{Insult Section:}
\begin{itemize}
\item{The percentage of times incorrect choice is picked}
\item{The percentage of times no choice is picked}
\item{The percentage of times an incorrect key is pressed}
\end{itemize}

After the user is done playing, they will be given the following statements as Likert scale questions, along with space to provide reasoning for their answers:

\begin{itemize}
\item{The game mechanics were easy to understand}
\item{The game was too difficult}
\item{There were too many distractions which hindered my abilities}
\item{The opponent seemed to have an advantage during the fight}
\item{The game was enjoyable to play}
\item{When I was losing a battle, it felt impossible to regain momentum}
\item{When I was winning, it felt like it was impossible for me to lose}
\end{itemize}

\subsection{Movement Controls}
The movement controls of the player will have to be tested to ensure they do not feel sluggish and do not frustrate the player. The user will be asked to look around with the mouse, move forward, backwards, strafe left and right, jump, and sprint to exhaust all possible character movements.

After the user is done performing these actions, they will be given the following statements as Likert scale questions, along with space to provide reasoning for their answers:

\begin{itemize}
\item{The game's controls felt responsive}
\item{The game's controls felt sluggish}
\item{The sprint was not fast enough}
\item{The jump was too high}
\end{itemize}